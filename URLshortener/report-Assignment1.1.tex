
\documentclass[a4paper]{article}
 \usepackage{indentfirst}
\usepackage[margin=1.9cm, top= 10mm]{geometry}

\begin{document}
\title{Assignment 1.2: Web Service (REST)}
\author{
  \textbf{Bernardo Rui Andrade}\\
  12531650 \\
  \texttt{bernardo.andrade@tecnico.ulisboa.pt} \\
  \and
  \textbf{João Almeida}\\
 12531693\\
  \texttt{joao.santos.almeida@tecnico.ulisboa.pt} \\
}
\maketitle
\section*{Design and implementation description of the URL shortener}
The architecture of our Web Service is based on a Client-Server model. In this case, the service is stateless meaning that the client will only be allowed to make independent requests to the server. The client reaches the server through a hardcoded url and has a previous knowledge of the available functions.  .\par
The REST Web Service was implemented in java using Spring-boot framework and Maven. The server consists in three parts: URLshortener, Controller and Application.\par
The controller is responsible to answer to the remote REST requests.  It was implemented using spring-boot and has a \textit{@RestController} annotation such that it can be treated as one by spring-boot. In the controller are every method that the server provides. Every method has the annotation \textit{@RequestMapping} that maps the URL and the type of the HTTP request to the proper method that will be responsible to handle the request. Then, the method calls a domain function in the URLshortener to treat the request, and depending on the response from the URLshortener sends the appropriate reply (success or error). \par
The URLshortener is a singleton java class that handles all the domain functions. This way, this instance can be shared between multiple clients if necessary and all the clients will have the same map URL to ID. It provides all the methods that can be called by the Controller and maintains a hashmap that maps URLs to IDs.\par
The Application is just a simple class with \textit{@SpringBootApplication} annotation to tell sprinb-boot where is the main class to start the application. \par
In the client there is only the application with the main method that is responsible to execute a variety of HTTP request to the server that will test every type of response for every availabe function that the server provides. 

\section*{Solution for making a URLshortener for multiple users}
A simple solution for multiple users would be adapting this solution to control simultaneous access to the same variables (race condtitions) and create a database that would save all the pairs URL - ID. It is also necessary to implement an interface that diferent users can call to use all the methods available. The way that the url are being shortened must be review to allow a better scalabiltiy, right now is just a counter that increments for each new URL that arrives to the system. A new way of shortening based in some hash function that would still shrink the URL would be a better solution for a large amount of users. 

 
\end{document}